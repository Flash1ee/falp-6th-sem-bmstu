\section*{1.Напишите функцию, которая уменьшает на 10 все числа из списка-аргумента этой функции}

\lstinputlisting[
	caption={Задание 1},
	label={lst:t1},
	language=lisp,
	linerange={3-7},
	]{../src/main.lsp}


\section*{2. Напишите функцию, которая умножает на заданное число-аргумент все числа из заданного списка-аргумента, когда а) все элементы списка -- числа; б) элементы списка -- любые объекты}

\lstinputlisting[
	caption={Задание 2},
	label={lst:t2},
	language=lisp,
	linerange={10-20},
	]{../src/main.lsp}

\section*{3. Напишите функцию, которая по своему списку-аргументу lst определяет, является ли он палиндромом}

\lstinputlisting[
	caption={Задание 3},
	label={lst:t3},
	language=lisp,
	linerange={23-27},
	]{../src/main.lsp}


\section*{4. Написать предикат set-equal, который возвращает T, если два его множества-аргумента содержат одни и те же элементы, порядок которых не имеет значения}

\lstinputlisting[
	caption={Задание 4},
	label={lst:t4},
	language=lisp,
	linerange={30-42},
	]{../src/main.lsp}


\section*{5. Написать функцию, которая получает как аргумент список чисел а возвращает список квадратов этих чисел в том же порядке}

\lstinputlisting[
	caption={Задание 5},
	label={lst:t5},
	language=lisp,,
	linerange={45-48},
	]{../src/main.lsp}


\section*{6. Напишите функцию select-between, которая из списка-аргумента из 5 чисел выбирает те, которые расположены между двумя указанными границами-аргументами и возвращает их в виде списка (сортированного +2 балла)}

\lstinputlisting[
	caption={Задание 6},
	label={lst:t6},
	language=lisp,
	linerange={51-60},
	]{../src/main.lsp}

\section*{7. Напишите функцию, вычисляющую декартово произведение двух своих списков-аргументов}

\lstinputlisting[
	caption={Задание 7},
	label={lst:t7},
	language=lisp,
	linerange={63-70},
	]{../src/main.lsp}
	
\section*{8. Почему так реализовано reduce, в чем причина?}
\begin{lstlisting}
(reduce #'+ ())
; > 0
; Так как функция + возвращает 0 при количестве аргументов 0
; reduce сработает для пустого списка
; Если бы функция не могла принимать 0 аргументов, то была бы ошибка
(reduce #'/ ())
; invalid number of arguments: 0
;    [Condition of type SB-INT:SIMPLE-PROGRAM-ERROR]

\end{lstlisting}

\section*{9. Написать функцию, которая вычисляет сумму длин всех элементов}

\lstinputlisting[
	caption={Задание 9},
	label={lst:t9},
	language=lisp,
	linerange={72-77},
	]{../src/main.lsp}

