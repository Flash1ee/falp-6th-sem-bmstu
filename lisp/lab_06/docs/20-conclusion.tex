% \chapter*{Заключение}
% \addcontentsline{toc}{chapter}{Заключение}

% В ходе выполнения работы были выполнены все поставленные задачи и изучены методы динамического программирования на основе алгоритмов вычисления расстояния Левенштейна.

% С помощью экспериментов были установлены различия в производительности алгоритмов вычисления расстояния Левенштейна. Для слов длины 10 рекурсивный алгоритм Левенштейна работает на несколько порядков медленнее (20000 раз) матричной реализации. Рекурсивный алгоритм с кешированием работает быстрее простого рекурсивного, но все еще медленнее матричного (150 раз). Если длина сравниваемых строк превышает 10, рекурсивный алгоритм становится неприемлимым для использования по времени выполнения программы. Матричная реализация алгоритма Дамерау — Левенштейна сопоставимо с алгоритмом Левенштейна. В ней добавлены дополнительные проверки, но, эти алгоритмы находятся в разном поле использования.

% Теоретически было рассчитано использования памяти в каждом из алгоритмов вычисления расстояния Левенштейна. Обычные матричные алгоритмы потребляют намного больше памяти, чем рекурсивные, за счет дополнительного выделения памяти под матрицы и большее количество локальных переменных, однако позволяют обрабатывать более длинные строки.