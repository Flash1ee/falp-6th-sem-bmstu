\chapter*{Введение}
\addcontentsline{toc}{chapter}{Введение}

Цель лабораторной работы  - изучение, реализация и исследование алгоритмов нахождения расстояний Левенштейна и Дамерау -- Левенштейна.


Расстояние Левенштейна (редакционное расстояние) — метрика, позволяющая определить минимальное количество операций вставки одного символа, удаления одного символа и замены одного символа на другой, необходимых для превращения одной строки в другую \cite{editor_distance}. Операциям, используемым в этом преобразовании, можно назначить разные цены. Широко используется в биоинформатике и компьютерной лингвистике.


Впервые задачу поставил в 1965 году советский математик Владимир Левенштейн при изучении последовательностей 0--1, впоследствии более общую задачу для произвольного алфавита связали с его именем \cite{Levenshtein}.


Расстояние Левенштейна применяется: 
\begin{enumerate}[label={\arabic*)}]
	\item для исправления ошибок в слове (в поисковых системах, базах данных, при вводе текста, при автоматическом распознавании отсканированного текста или речи);
	\item для сравнения текстовых файлов утилитой \code{diff} unix систем (здесь роль «символов» играют строки, а роль «строк» — файлы);
	\item в биоинформатике для сравнения генов, белков и др..
\end{enumerate}


Расстояние Дамерау — Левенштейна (названо в честь учёных Фредерика Дамерау и Владимира Левенштейна) — этот алгоритм является модификацией расстояния Левенштейна. В нём к операциям удаления, вставки, замены добавляется операция транспозиции (перестановки двух соседних символов). 

\vspace{2mm}
Задачи лабораторной работы:
\begin{itemize}
	\item изучение алгоритмов Левенштейна и Дамерау--Левенштейна;
	\item получение практических навыков реализации алгоритмов Левенштейна и Дамерау — Левенштейна;
	\item исследование алгоритмов;
	\item применение методов динамического программирования;
	\item проведение сравнительного анализа алгоритмов на основе полученных экспериментальных данных;
	\item сравнительный анализ алгоритмов на основе экспериментальных данных;
	\item подготовка отчета по лабораторной работе.
\end{itemize}
