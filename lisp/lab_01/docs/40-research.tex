\chapter{Исследовательская часть}

\section{Пример работы}

Программа имеет консольный интерфейс взаимодействия. Демонстрация работы программы приведена на рисунках
\ref{img:demo_levenshtein}, \ref{img:demo_damerau_iter}, \ref{img:demo_damerau_rec}.
\boximg{50mm}{demo_levenshtein}{Демонстрация работы алгоритмов расстояния Левенштейна}
\boximg{35mm}{demo_damerau_iter}{Демонстрация работы алгоритма нахождения расстояния Дамерау -- Левенштейна с кэшированием}
\boximg{35mm}{demo_damerau_rec}{Демонстрация работы алгоритма нахождения расстояния Дамерау -- Левенштейна рекурсивно}
\clearpage

\section{Технические характеристики}

Технические характеристики устройства, на котором выполнялось тестирование:

\begin{itemize}
	\item Операционная система: Ubuntu 21.04 \cite{ubuntu} Linux \cite{linux} 64-bit.
	\item Память: 19.4 GiB.
	\item Процессор: Intel Core™ i5-8300H \cite{intel} CPU @ 2.30GHz 
\end{itemize}

Тестирование проводилось на ноутбуке, включенном в сеть электропитания. Во время тестирования ноутбук был нагружен только системой тестирования и системным окружением операционной системы.

\section{Время выполнения алгоритмов}

Алгоритмы тестировались при помощи написания <<бенчмарков>> \cite{gotest}, предоставялемых встроенными в Go средствами. Для такого рода тестирования не нужно самостоятельно указывать количество повторов. В библиотеке для тестирования существует константа $N$, которая динамически настраивается, в зависимости от того, стабильный получен результат или нет.

В листинге \ref{lst:bench} пример реализации бенчмарка.
\begin{lstlisting}[label=lst:bench, style=go]

package levenshtein

import (
	"github.com/stretchr/testify/assert"
	"strings"
	"testing"
)

func TestGetDamerauLevenshteinIterative(t *testing.T) {
	var tests = []struct {
		fWord    string
		sWord    string
		expected int
	}{
		{expected: 0},
		{
			fWord:    "mama",
			sWord:    "papa",
			expected: 2,
		},
		{
			fWord:    "mama",
			sWord:    "mama",
			expected: 0,
		},
	}
	for _, pair := range tests {
		t.Run(strings.Join([]string{pair.fWord, pair.sWord}, " "), func(t *testing.T) {
			res := GetDamerauLevenshteinIterative(pair.fWord, pair.sWord)
			assert.Equal(t, pair.expected, res)
		})
	}
}
\end{lstlisting}
%\begin{lstinputlisting}[
%	caption={Реализация бенчмарка},
%	label={lst:levenshtein_bench},
%	style={go},
%	]{../src/levenshtein/levenshtein_test.go}
%\end{lstinputlisting}
%\begin{lstinputlisting}[
%	caption={Бенчмарк},
%	label={lst:bench},
%	style=go,
%	linerange={1-39}
%]{../src/levenshtein/levenshtein_test.go}
%\end{lstinputlisting}


Результаты замеров приведены в таблице \ref{tbl:time}. Отсутствие времени в таблице означает, что тестирование для данных значений не выполнялось.
Два случая: либо тестирование выполняется быстро, не удаётся получить результат в нс, либо очень долго.
На рисунках \ref{plt:time_levenshtein} и \ref{plt:time_dl} приведены графики зависимостей времени работы алгоритмов от длины строк.

\begin{table}[h]
	\begin{center}
		\caption{Замер времени для строк, размером от 5 до 50}
		\label{tbl:time}
		\begin{tabular}{|c|c|c|c|c|c|}
			\hline
			                      & \multicolumn{5}{c|}{\bfseries Время, нс}                                    \\ \cline{2-5}
			\bfseries Длина строк & \bfseries LevRec & \bfseries LevRecCache & \bfseries LevMat & \bfseries DLRec & \bfseries DLMat
			\csvreader{inc/csv/benchmark.csv}{}
			{\\\hline \csvcoli&\csvcolii&\csvcoliii&\csvcoliv&\csvcolv&\csvcoliv}
			\\\hline
		\end{tabular}
	\end{center}
\end{table}


\begin{figure}[h]
	\centering
	\begin{tikzpicture}
		\begin{axis}[
			axis lines=left,
			xlabel=Длина строк в символах,
			ylabel={Время, нс},
			legend pos=north west,
			ymajorgrids=true
		]
			\addplot table[x=len,y=RecursiveMatrix,col sep=comma] {inc/csv/bench_recmat.csv};
			\addplot table[x=len,y=IterativeMatrix,col sep=comma] {inc/csv/bench_itmat.csv};
			\legend{Рекурсивный с матрицей, Матричный}
		\end{axis}
	\end{tikzpicture}
	\captionsetup{justification=centering}
	\caption{Зависимость времени работы алгоритма вычисления расстояния Левенштейна от длины строк (рекурсивная с кэшем и матричная реализации)}
\label{plt:time_levenshtein}
\end{figure}

\begin{figure}[h]
	\centering
	\begin{tikzpicture}
		\begin{axis}[
			axis lines=left,
			xlabel={Длина строк в символах},
			ylabel={Время, нс},
			legend pos=north west,
			ymajorgrids=true
		]
			\addplot table[x=len,y=IterativeMatrix,col sep=comma] {inc/csv/bench_itmat.csv};
			\addplot table[x=len,y=DamerauLevenshtein,col sep=comma] {inc/csv/bench_dl.csv};
			\legend{Левенштейн, Д. — Левенштейн}
		\end{axis}
	\end{tikzpicture}
	\captionsetup{justification=centering}
	\caption{Зависимость времени работы матричных реализаций алгоритмов Левенштейна и Дамерау — Левенштейна}
	\label{plt:time_dl}
\end{figure}


\section{Использование памяти}

Алгоритмы Левенштейна и Дамерау — Левенштейна не отличаются друг от друга с точки зрения использования памяти.
Имеет смысл рассмотреть лишь разницу рекурсивной и матричной реализаций этих алгоритмов.

Максимальная глубина стека вызовов при рекурсивной реализации равна сумме длин входящих строк, максимальный расход памяти (\ref{for:rec})
\begin{equation}
(\mathcal{L}(S_1) + \mathcal{L}(S_2)) \cdot (2 \cdot \mathcal{L}\mathrm{(string)} + 3 \cdot \mathcal{L}\mathrm{(integer)}),
\label{for:rec}
\end{equation}
где $\mathcal{L}$ — оператор вычисления размера, $S_1$, $S_2$ — строки, $\mathrm{integer}$ — целочисленный тип, $\mathrm{string}$ — строковый тип.

Использование памяти при итеративной реализации теоретически равно
\begin{equation}
(\mathcal{L}(S_1) + 1) \cdot (\mathcal{L}(S_2) + 1) \cdot \mathcal{L}\mathrm{(integer)} + 7\cdot \mathcal{L}\mathrm{(integer)} + 2 \cdot \mathcal{L}\mathrm{(string)}.
\end{equation}

\section{Вывод}

Рекурсивный алгоритм Левенштейна работает на порядок дольше итеративных реализаций, время его работы увеличивается в геометрической прогрессии. На словах длиной 10 символов, матричная реализация алгоритма Левенштейна превосходит по времени работы рекурсивную в 20000 раз. Рекурсивный алгоритм с заполнением матрицы превосходит простой рекурсивный на аналогичных данных в 150 раз. Алгоритм Дамерау — Левенштейна по времени выполнения сопоставим с алгоритмом Левенштейна. В нём добавлены дополнительные проверки, и по сути он является алгоритмом другого смыслового уровня.

Но по расходу памяти итеративные алгоритмы проигрывают рекурсивному: максимальный размер используемой памяти в них растёт как произведение длин строк, в то время как у рекурсивного алгоритма — как сумма длин строк.
