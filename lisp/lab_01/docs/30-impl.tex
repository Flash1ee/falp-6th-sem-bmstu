\chapter{Технологическая часть}

В данном разделе приведены требования к программному обеспечению, средства реализации и листинги кода.

\section{Требования к ПО}

К программе предъявляются следующие требования:
\begin{itemize}
	\item на вход подаётся две строки и флаги, запускающие одну из реализаций алгоритма или тестирование;
	\item на выходе — искомое расстояние для одного (или четырех) методов.
\end{itemize}

\section{Средства реализации}

В качестве языка программирования для реализации лабораторной работы был выбран язык Golang \cite{golang}. Данный выбор обусловлен моим желанием получить практику его применения, а также наличием встроенных средств, облегчающих процесс тестирования приложения.

\section{Листинг кода}

В листингах \ref{lst:levenshtein} --\ref{lst:levenshtein_utils} приведены реализации алгоритмов Левенштейна и Дамерау — Левенштейна, а также вспомогательные функции.

\begin{lstinputlisting}[
	caption={Реализация алгоритмов Левенштейна},
	label={lst:levenshtein},
	style=go
]{../src/levenshtein/levenshtein.go}
\end{lstinputlisting}

\begin{lstinputlisting}[
	caption={Реализация алгоритмов Дамерау -- Левенштейна},
	label={lst:damerau_levenshtein},
	style=go
]{../src/levenshtein/damerau-levenshtein.go}
\end{lstinputlisting}

\begin{lstinputlisting}[
	caption={Вспомогательные функции и типы данных},
	label={lst:levenshtein_utils},
	style=go
	]{../src/levenshtein/utils.go}
\end{lstinputlisting}

\clearpage

В таблице \ref{tabular:functional_test} приведены функциональные тесты для алгоритмов вычисления расстояния Левенштейна и Дамерау — Левенштейна. Все тесты пройдены успешно.

\begin{table}[h]
	\begin{center}
		\caption{\label{tabular:functional_test} Функциональные тесты}
		\begin{tabular}{|c|c|c|c|}
			\hline
			&& \multicolumn{2}{c|}{\bfseries Ожидаемый результат}\\ \cline{3-4}
			\bfseries Строка 1 & \bfseries Строка 2 & \bfseries Левенштейн & \bfseries Дамерау — Левенштейн\\
			\hline
			mama&	papa&	2&	2\\
			\hline
			mama & mama & 0& 0\\
			\hline
			kot & skat & 2 & 2\\
			\hline
			apple & aplpe & 2 & 1\\
			\hline
			kot & "" & 3 & 3\\
			\hline
			"" & "" & 0 & 0\\
			\hline
			qwerty & ytrewq & 6 & 5\\
			\hline
		\end{tabular}
	\end{center}
\end{table}


\section{Вывод}

Были реализованы и протестированы спроектированные алгоритмы: вычисления расстояния Левенштейна рекурсивно, с заполнением матрицы, рекурсивно с заполнением матрицы, а также вычисления расстояния Дамерау — Левенштейна с заполнением матрицы.
Были протестированы алгоритмы на входных данных, покрывающих граничные значения (пустые строки, комбинации пустых с непустыми) и обычные случаи (равные по длине строки, неравные, равные по символам, неравные).
