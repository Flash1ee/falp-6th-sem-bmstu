\section*{Задание 1}
\subsection*{Постановка задачи}

Составить диаграмму вычисления следующих выражений:

\begin{enumerate}
	\item (equal 3 (abs -3))
	\item (equal (+ 1 2) 3)
	\item (equal (* 4 7) 21)
	\item (equal (* 2 3) (+ 7 2))
	\item (equal (- 7 3) (* 3 2)))
	\item (equal (abs (- 2 4)) 3)
\end{enumerate}

\subsection*{Решение}
\lstinputlisting[language=lisp]{../src/1.lsp}

\section*{Задание 2}
\subsection*{Постановка задачи}

Написать функцию, вычисляющую гипотенузу прямоугольного треугольника по заданным катетам и составить диаграмму ее вычисления. Решение.

\subsection*{Решение}
\lstinputlisting[language=lisp]{../src/2.lsp}


\section*{Задание 3}
\subsection*{Условие задачи}
Написать функцию, вычисляющую объем параллелепипеда по 3-м его сторонам, и составить диаграмму ее вычисления.

\subsection*{Решение}
\lstinputlisting[language=lisp]{../src/3.lsp}

\section*{Задание 4}
\subsection*{Условие задачи}
Каковы результаты вычисления следующих выражений? (объяснить возможную ошибку и
варианты ее устранения)

\subsection*{Решение}

\begin{lstlisting}[label=4xd,caption=Решение задания №4, language=lisp]
(list 'a c); THE VARIABLE C IS UNBOUND; (list `a `c) -> (AC)
(cons `a (b c)); THE VARIABLE C IS UNBOUND; (cons `a `(bc)) -> (ABC)
(cons `a `(b c)) -> (A B C)
(caddy (1 2 3 4 5)) -> (caddr `(1 2 3 4 5)) -> 4
(cons `a `b `c); INVALID NUMBER OF ARGUMENTS; (cons `a `b) -> (A . B)
(list `a (b c)); THE VARIABLE C IS UNBOUND; (list `a `(b c)) -> (A (BC))
(list a `(b c)); THE VARIABLE A IS UNBOUND; (list `a `(b c)) -> (A (BC))
(list (+ 1 `(length `(1 2 3)))) ; (LENGTH `(1 2 3)) is not of type NUMBER; (list (+1 (length `(123)))) -> 4
\end{lstlisting}

\section*{Задание 5}
\subsection*{Условие задачи}
Написать функцию \textbf{longer\_then} от двух списков- аргументов, которая возвращает T, если первый аргумент имеет большую длину.

\subsection*{Решение}

\begin{lstlisting}[label=5xd,caption=Решение задания №5, language=lisp]
(defun longer_than (l1 l2) (> (length l1) (length l2)))
\end{lstlisting}

\section*{Задание 6}
\subsection*{Условие задачи}
Каковы результаты вычисления следующих выражений? 

\subsection*{Решение}

\begin{lstlisting}[label=6xd,caption=Решение задания №6, language=lisp]
(cons 3 (list 5 6)) -> (356)
(cons 3 `(list 5 6)) -> (3 LIST 5 6)
(list 3 'from 9 'lives (- 9 3)) -> 3 FROM 9 LIVES 6
(+ (length for 2 too)) (car '(21 22 23))) -> error
(cdr `(cons is short for ans)) -> (IS SHORT FOR ANS)
(car (list one two)); VARIABLE ONE IS UNBOUND; (car (list `one `two)); -> ONE
(car (list `one `two)) -> ONE
\end{lstlisting}

\section*{Задание 7}
\subsection*{Условие задачи}
Дана функция \textbf{(defun mystery (x) (list (second x) (first x)))}. Какие результаты вычисления следующих выражений?

\subsection*{Решение}

\begin{lstlisting}[label=7xd,caption=Решение задания №7, language=lisp]
(mystery (one two)) -> error
(mystery free); error (mystery `(free)) -> (NIL FREE)
(mystery (last one two)); error; (mystery (last (`one `two))) -> (NIL TWO)
(mystery one `two); error (mystery `(one two)) -> (TWO ONE)
\end{lstlisting}

\section*{Задание 8}
\subsection*{Условие задачи}
Написать функцию, которая переводит температуру в системе Фаренгейта в  
температуру по Цельсию (defum f-to-c (temp)…).  

Формулы:  
c = 5/9*(f-320); f= 9/5*c+32.0.  

Как бы назывался роман Р.Брэдбери "+451 по Фаренгейту" в системе по Цельсию?  

\subsection*{Решение}
\lstinputlisting[label=7xd,caption=Решение задания №8, language=lisp]{../src/8.lsp}

\section*{Задание 9}
\subsection*{Условие задачи}
Что получится при вычисления каждого из выражений?

\subsection*{Решение}
\begin{lstlisting}[label=7xd,caption=Решение задания №9, language=lisp]
(list 'cons t NIL); (CONS T NIL)
(eval (list 'cons t NIL)) ; (T)
(eval (eval (list 'cons t NIL))) error
(apply #cons "(t NIL)) 
(eval NIL)
(list 'eval NIL) 
(eval (list 'eval NIL))
\end{lstlisting}

\section*{Контрольные вопросы}

\textbf{Вопрос 1.} Базис языка Lisp. \newline
\indent\textbf{Ответ. }
Базис языка представлен:
\begin{itemize}
	\item структурами и атомами;
	\item функциями;
\end{itemize}

Функции, входящие в базис языка:
\begin{itemize}
	\item atom, eq, cons, car, cdr;
	\item cond, quote, lambda, eval, label.
\end{itemize}


\textbf{Вопрос 2.} Классификация функций языка Lisp.
	
\textbf{Ответ.} 
	
\begin{itemize}
	\item чистые (с фиксированным количеством аргументов) математические функции;
	\item рекурсивные функции;
	\item специальные функции – формы (принимают произвольное количество аргументов или по разному обрабатывают аргументы);
	\item псевдофункции (создающие «эффект» – отображающие на экране процесс обработки данных и т.п.);
	\item функции с вариативными значениями, выбирающие одно значение;
	\item функции высших порядков – функционалы (используются для построения синтаксически управляемых программ);
\end{itemize}

\textbf{Вопрос 3.} Синтаксис элементов языка и их представление в памяти.


\textbf{Ответ.}""\newline


Точечные пары ::= (<атом>, <атом>) |

(<атом>, <точечная пара>) |

(<точечная пара>, <атом>) |

(<точечная пара>, <точечная пара>)""\newline

\indent Список ::= <пустой список> | <непустой список>, где

<пустой список> ::= () | Nil,

<непустой список> ::= (<первый элемент>, <хвост>),

<первый элемент> ::= <S-выражение>,

<хвост> ::= <список>""\newline


\indent \textbf{Список} -- частный случай S-выражения. Любая структура (точечная пара или список) заключаются в круглые скобки:


\begin{itemize}

	\item (A . B) -- точечная пара;

	\item $(A)$ -- список из одного элемента;

	\item $Nil$ или $()$ -- пустой список;

	\item (A . (B . (C . (D ()))))) или (A B C D) -- непустой список;

	\item Элементы списка могу являться списками: $((A)(B)(CD))$

\end{itemize}


Любая непустая структура в Lisp, в памяти представленна списковой ячейкой, хранящей два указателя: на голову и хвост.""\newline

\textbf{Вопрос 4.} Функции \textbf{car}, \textbf{cdr}.
	
\textbf{Ответ.} Функции $car$, $cdr$ являются базовыми функциями доступа к
данным. car принимает точечную пару или список в качестве аргумента
и возвращает первый элемент или $Nil$, соответственно. $cdr$ принимает точечную пару или список в качестве аргумента и возвращает все элементы
кроме первого или $Nil$, соответственно.""\newline
	
\textbf{Вопрос 5.} Функции \textbf{list}, \textbf{cons}.
	
\textbf{Ответ.} Функции $list$, $cons$ являются функциями создания списков
($cons$ – базовая, $list$ – нет). $cons$ создает списочную ячейку и устанавливает два указателя на аргументы. $list$ принимает переменное число аргументов и возвращает список, элементы которого – переданные в функцию
аргументы.