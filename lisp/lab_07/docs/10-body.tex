\section*{1.Написать хвостовую рекурсивную функцию my-reverse, которая развернет верхний уровень списка-аргумента lst}

\lstinputlisting[
	caption={Задание 1},
	label={lst:t1},
	language=lisp,
	linerange={1-6},
	]{../src/main.lsp}


\section*{2.Написать функцию, которая возвращает первый элемент списка-аргумента, который сам является непустым списком}

\lstinputlisting[
	caption={Задание 2},
	label={lst:t2},
	language=lisp,
	linerange={8-14},
	]{../src/main.lsp}

\section*{3.Написать функцию, которая выбирает из заданного списка только те числа, которые больше 1 и меньше 10 или нефиксированные}

\lstinputlisting[
	caption={Задание 3},
	label={lst:t3},
	language=lisp,
	linerange={16-21},
	]{../src/main.lsp}


\section*{4. Напишите рекурсивную функцию, которая умножает на заданное число-аргумент все числа а) все элементы списка -- числа б) элементы списка -- любые объекты}

\lstinputlisting[
	caption={Задание 4},
	label={lst:t4},
	language=lisp,
	linerange={24-37},
	]{../src/main.lsp}


\section*{5. Напишите функцию select-between, которая из списка-аргумента из чисел выбирает те, которые расположены между двумя указанными границами-аргументами и возвращает их в виде списка (сортированного +2 балла)}

\lstinputlisting[
	caption={Задание 5},
	label={lst:t5},
	language=lisp,,
	linerange={40-48},
	]{../src/main.lsp}


\section*{6. Написать рекурсивную версию (с именем rec-add) вычисления суммы чисел заданного списка: а) одноуровнего смешанного б) структурированного}

\lstinputlisting[
	caption={Задание 6},
	label={lst:t6},
	language=lisp,
	linerange={56-71},
	]{../src/main.lsp}

\section*{7. Написать рекурсивную версию с именем recnth функции nth}
\lstinputlisting[
	caption={Задание 7},
	label={lst:t7},
	language=lisp,
	linerange={74-85},
	]{../src/main.lsp}
	
\section*{8. Написать рекурсивную функцию allodd, которая возвращает t когда все элементы списка нечетные}
\lstinputlisting[
	caption={Задание 9},
	label={lst:t9},
	language=lisp,
	linerange={88-94},
	]{../src/main.lsp}


\section*{9. Написать рекурсивную функцию, которая возвращает первое нечетное число из списка (структурированного), возможно создавая некоторые вспомогательные функции.}

\lstinputlisting[
	caption={Задание 9},
	label={lst:t9},
	language=lisp,
	linerange={97-104},
	]{../src/main.lsp}

\section*{10. Используя cons-дополняемую рекурсию с одним тестом завершения, написать функцию которая получает как аргумент список чисел, а возвращает список квадратов этих чисел в том же порядке.}

\lstinputlisting[
	caption={Задание 10},
	label={lst:t10},
	language=lisp,
	linerange={107-110},
	]{../src/main.lsp}
