\section*{Задание 1}
\section*{Постановка задачи}

Составить программу — базу знаний, с помощью которой можно определить, например, множество студентов, обучающихся в одном ВУЗе. Студент может одновременно обучаться в нескольких ВУЗах. Привести примеры возможных вариантов вопросов и варианты ответов (не менее 3-х), Описать порядок формирования вариантов ответа.

\subsection*{Решение}
\begin{lstlisting}
domains
    id = unsigned.
    name, university = symbol.

predicates
    StudentID(id, name).
    StudyIn(id, university).
    studentsFromUniversity(university, id, name).

clauses
    StudentID(0, "Dmitriy").
    StudentID(1, "Dmitriy").
    StudentID(2, "Volodya").
    StudentID(3, "Maxim").
    StudentID(4, "Nikolay").

    StudyIn(0, "BMSTU").
    StudyIn(1, "MSU").
    StudyIn(2, "MAI").
    StudyIn(3, "BMSTU").
    StudyIn(4, "BMSTU").
    StudyIn(0, "KTU").
    StudyIn(2, "Cambridge").
    StudyIn(3, "MIREA").

    studentsFromUniversity(University, Id, Name) :- StudentID(Id, Name), StudyIn(Id, University).

goal
    StudentID(ID, "Dmitriy").
    % StudentID(0, Name).

    % StudyIn(ID, "BMSTU").
    % StudyIn(0, U).

    % studentsFromUniversity("BMSTU", ID, NAME).
    % studentsFromUniversity(U, ID, "Maxim").
\end{lstlisting}

Данная база знаний содержит информацию о студентах (номер, имя, вуз).\\

С помощью первого вопроса получаются все идентификаторы студентов, имя которых Dmitriy. Происходит проход сверху вниз по всем фактам
предиката StudentID(id, name) и осуществляется унификация с
StudentID(ID, "Dmiitriy"). Унификацию успешно проходит два факта:
StudentID(0, "Dmitriy") и StudentID(1, "Dmitriy").

\\

С помощью третьего вопроса получаются все студенты, которые обучаются в МГТУ. Происходит проход по всем фактам предиката StudyIn(ID, university) и осуществляется унификация с StudyIn(ID, "BMSTU"). Успешно унификацию проходят факты StudyIn(0, "BMSTU"), StudyIn(1,
"BMSTU"), StudyIn(3, "BMSTU"), StudyIn(4, "BMSTU").  

\\

С помощью шестого вопроса получаются все университеты и идентификаторы учащихся, имя которых Maxim. Происходит проход по всем фактам
предикатов StudentID(id, name), StudyIn(id, university) и осуществляется унификация с studentsFromUniversity(University, Id, "Maxim"). Успешно унификацию проходят правила StudentID(0, "Maxim"), StudyIn(0,
"BMSTU") и StudentID(0, "Maxim"), studyIn(0, "MiREA").