\section*{Задание 1}
 \subsection*{Постановка задачи}
\textbf{Задание:} Создать базу знаний «Собственники», дополнив (и минимально изменив) базу
знаний, хранящую знания (лаб. 12):
\begin{itemize}
    \item \textbf{<<Телефонный справочник>>:} Фамилия, №тел, Адрес - структура (Город, Улица, №дома, №кв),
    \item \textbf{<<Автомобили>>:} Фамилия\_владельца, Марка, Цвет, Стоимость, и др.,
    \item \textbf{<<Вкладчики банка>>:} Фамилия, Банк, счет, сумма, др.
\end{itemize}
знаниями о дополнительной собственности владельца. Преобразовать знания об
автомобиле к форме знаний о собственности.

Вид собственности (кроме автомобиля):
\begin{itemize}
\item  \textbf{Строение, стоимость} и другие его характеристики;
\item  \textbf{Участок, стоимость} и другие его характеристики;
\item  \textbf{Водный транспорт, стоимость} и другие его характеристики.
\end{itemize}

Описать и использовать вариантный домен: \textbf{Собственность}. Владелец может иметь,
но  \textbf{только один} объект  \textbf{каждого вида собственности} (это касается и  \textbf{автомобиля}), или не иметь некоторых видов собственности. 

Используя  \textbf{конъюнктивное правило и
разные формы} задания  \textbf{одного вопроса} (\textbf{пояснять} для какого №задания – какой вопрос),
обеспечить возможность поиска:
\begin{enumerate}
\item Названий всех объектов собственности заданного субъекта;
\item Названий и стоимости всех объектов собственности заданного субъекта;
\item *Разработать правило, позволяющее найти суммарную стоимость всех
объектов собственности заданного субъекта.
\end{enumerate}

Для 2-го пункт и  \textbf{одной} фамилии  \textbf{составить таблицу}, отражающую конкретный
порядок работы системы, с объяснениями порядка работы и особенностей использования
доменов (указать конкретные Т1 и Т2 и полную подстановку на каждом шаге)
\includegraphics[scale=0.6]{./inc/img/tb_tmpl}


\clearpage
\subsection*{Решение}
\lstinputlisting[
	caption={Задание 1},
	label={lst:t1},
	language=lisp,
	]{../src/main.pro}