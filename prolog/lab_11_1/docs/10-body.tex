\section*{Задание 1}
\subsection*{Постановка задачи}

\textbf{Цель работы} --- познакомиться со средой Visual Prolog, познакомиться со структурой программы: способом запуска и формой вывода результатов.  

\textbf{Задачи работы} --- изучить принципы работы в среде VisualProlog, возможность получения однократного и многократного результата, изучить базовые конструкции языка Prolog, структуру програмым Prolog, форму ввода исходных данных и вывода результатов работы программы.

\subsection*{Теоретические вопросы}
1. Что собой представляет программа на Prolog? 

Программа на Prolog представляет из себя базу знаний, состоящую из фактов и теорем, которые задаются с помощью терм, являющихся логическими переменными.

Структура программы на Prolog - Программа на Prolog состоит из разделов. Каждый раздел начинается со своего заголовка. 

Структура программы:\\
Директивы компилятора —— зарезервизарезервирроованные символьные константы.

\begin{itemize}
\item CONSTANTS —— раздел описания констант;
\item DOMAINS —— раздел описания доменов;
\item DATABASE —— раздел описания предикатов внутренней базы данных;
\item PREDICATES —— раздел описания предикатов;
\item CLAUSES —— раздел описания предложений базы знаний;
\item GOAL —— раздел описания внутренней цели (вопроса).
\end{itemize}

В программе не обязательно должны быть все разделы.\\

2. Каковы результаты работы программы? 

В качестве результата Prolog возвращает ответ на вопрос в виде ''да'' или ''нет'' и если да, то также все значения переменных, которые позволили получить ответ ''да'', если переменные использовались в вопросе.\\


\chapter{Практические задания}
\section*{Задание 1}
Разработать программу - "Телефонный справочник". Протестировать работу программы.

\begin{lstlisting}
domains
	name = string
	city = string
	phone = integer

predicates
  	record(name, city, phone)
  
clauses
	record("Dmitry", "Moscow", "8(953)827-23-16").
	record("Alexey", "Novgorod", "8(960)522-74-27").
	record("Kate", "Kiev", "8(127)76-03-11").
	record("Vasily", "Krasnodar", "8(720)311-38-45").
	record("Maxim", "Samara", "8(495)641-49-14").
	record("Nastya", "London", "8(120)821-47-74").
	record("Volodya", "Kiev", "8(921)128-43-90").

goal
	record(Name, "Kiev", Phone).
\end{lstlisting}