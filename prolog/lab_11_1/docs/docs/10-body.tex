\section*{Задание 1}
\subsection*{Постановка задачи}

\textbf{Цель работы} --- познакомиться со средой Visual Prolog, познакомиться со структурой программы: способом запуска и формой вывода результатов.  

\textbf{Задачи работы} --- изучить принципы работы в среде VisualProlog, возможность получения однократного и многократного результата, изучить базовые конструкции языка Prolog, структуру програмым Prolog, форму ввода исходных данных и вывода результатов работы программы.

\subsection*{Теоретические вопросы}
\section*{1. Что собой представляет программа на языке пролог?}

Программа на Prolog представляет собой набор фактов и правил, обеспечивающих получение заключений на основе этих утверждений. Программа содержит базу знаний и вопрос. База знаний содержит истинные значения, используя которые программа выдает ответ на вопрос. 

Основным элементом языка является терм. База знаний состоит из предложений. Каждое предложение заканчивается точкой. Вопрос состоит только из тела – составного терма (или нескольких составных термов). Вопросы используются для выяснения выполнимости некоторого отношения между описанными в программе объектами. Система рассматривает вопрос как цель, к которой (к истинности которой) надо стремиться. Ответ на вопрос может оказаться логически положительным или отрицательным, в зависимости от того, может ли быть достигнута соответствующая цель.

\section*{2. Какова структура программы на Prolog?}

Программа на Prolog состоит из следующих разделов:

\begin{itemize}
	\item директивы компилятора — зарезервированные символьные константы,
	\item CONSTANTS — раздел описания констант,
	\item DOMAINS — раздел описания доменов,
	\item DATABASE — раздел описания предикатов внутренней базы данных,
	\item PREDICATES — раздел описания предикатов,
	\item CLAUSES — раздел описания предложений базы знаний,
	\item GOAL — раздел описания внутренней цели (вопроса).
\end{itemize}

В программе не обязательно должны быть все разделы.


\section*{3. Как реализуется программа на Prolog? Как формируются результаты работы программы?}

Ответ на поставленный вопрос система дает в логической форме - «Да» или «Нет». Цель системы состоит в том, чтобы на поставленный вопрос найти возможность, исходя из базы знаний, ответить «Да». Вариантов ответить «Да» на поставленный вопрос может быть несколько. В нашем случае система настроена в режим получения всех возможных вариантов ответа. При поиске ответов на вопрос рассматриваются альтернативные варианты и находятся все возможные решения (методом проб и ошибок) - множества значений переменных, при которых на поставленный вопрос можно ответить - «Да».

Для выполнения логического вывода используется механизм унификации, встроенный в систему.
Унификация – операция, которая позволяет формализовать процесс логического вывода. С практической точки зрения  - это основной вычислительный шаг, с помощью которого происходит:
\begin{itemize}
	\item Двунаправленная передача параметров процедурам,
	\item Неразрушающее присваивание,
	\item Проверка условий (доказательство).
\end{itemize}

В процессе работы система выполняет большое число унификаций.  Попытка "увидеть одинаковость" – сопоставимость двух термов, может завершаться успехом или тупиковой ситуацией (неудачей). В последнем случае включается механизм отката к предыдущему шагу.


\chapter{Практические задания}
\section*{Задание 1}
Разработать программу - "Телефонный справочник". Протестировать работу программы.

\begin{lstlisting}
domains
	name = string
	city = string
	phone = integer

predicates
  	record(name, city, phone)
  
clauses
	record("Dmitry", "Moscow", "8(953)827-23-16").
	record("Alexey", "Novgorod", "8(960)522-74-27").
	record("Kate", "Kiev", "8(127)76-03-11").
	record("Vasily", "Krasnodar", "8(720)311-38-45").
	record("Maxim", "Samara", "8(495)641-49-14").
	record("Nastya", "London", "8(120)821-47-74").
	record("Volodya", "Kiev", "8(921)128-43-90").

goal
	record(Name, "Kiev", Phone).
\end{lstlisting}